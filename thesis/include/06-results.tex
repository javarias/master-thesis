\chapter{Tests and results}

Two sets of tests will be performed to benchmark the algorithms proposed for both problems presented in this work.
The first set will show how the solution given to ``Astronomical observation scheduling problem'', focusing the comparison with current DSA implementation done for ALMA.
The second set will show the feasibility of the proposed algorithm to solve the ``Array configuration planning problem'', focusing in the results for different scenarios.

All the tests will be run on a 2 x Intel Xeon X5675 running at 3.06 $[GHz]$, with 48 $[GiB]$ of RAM DDR3 1333 $[MHz]$, configured in 6 channels (2 x QPI 6.4 $[GT/s]$). Running Red Hat Enterprise Linux 5.9 (x86 PAE) and Oracle's Java (JRE) 1.7.0\_51.

\section{Testing the algorithms for the Astronomical observation scheduling problem}

The main feature of the DRR algorithm, presented in this work, is to try to keep the observation time given to each Executive, according to the values presented in table~\ref{table:input-executive}, along the time array configuration life. For this reason, a comparison between the ALMA's DSA and the DRR will be done. 

The metric to compare both algorithm will be the percentage of time given to each executive every week, every two weeks, every four weeks and for the whole array configuration duration. At the same time, it will be analyzed the impact of the DRR algorithm for the project completion rate at the end of the whole simulation, checking if there is any adverse effect over this parameter. The completion rate will be based at project level, accounting projects completed, started but non-completed and non-started, these categories will be broken down based on the projects science grade given: $A$, $B$ and $C$.

The test to benchmark both algorithms will be based on the input data described in section~\ref{sec:input-data}. Although the array configuration planning will be fixed to the one presented in table~\ref{table:drr-test-array-config}
\begin{table}[h!]
\begin{center}
\begin{tabular}{|c|c|c|}
\hline
Array configuration & Start date (UTC) & End date (UTC) \\ \hline
C34-6 & 2014-06-01 & 2014-07-14 \\ \hline
C34-7 & 2014-07-15 & 2014-08-31 \\ \hline
C34-6 & 2014-12-01 & 2014-12-31 \\ \hline
C34-5 & 2015-01-01 & 2015-01-31 \\ \hline
C34-4 & 2015-03-01 & 2015-03-31 \\ \hline
C34-3 & 2015-04-01 & 2015-04-30 \\ \hline
C34-2 & 2015-05-01 & 2015-05-31 \\ \hline
C34-1 & 2015-06-01 & 2015-06-30 \\ \hline
C34-4 & 2015-07-01 & 2015-07-31 \\ \hline
C34-6 & 2015-08-01 & 2015-08-30 \\ \hline
C34-7 & 2015-09-01 & 2015-10-01 \\ \hline
\end{tabular}
\end{center}
\caption{Array configuration plan used to compare DRR algorithm vs DSA algorithm}
\label{table:drr-test-array-config}
\end{table}

The total amount of hours that the algorithm were able to schedule is summarized in table~\ref{table:dsa-hours-per-array} for the DSA execution, and in table~\ref{table:drr-hours-per-array}.

\begin{table}[h!]
\centering
\begin{tabular}{c|c|c|c|c|c|c|c|} 
\cline{2-8}
 & \multicolumn{7}{c|}{Executive observation time $[h]$} \\ \hline
\multicolumn{1}{|r|}{Array Type} & CL	& EA & EA\_NA &	EU & NA & OTHER & Total \\ \hline
\multicolumn{1}{|r|}{12m} & 186 & 444 & 90 & 804 & 786 & 42 & 2352 \\ \hline
\multicolumn{1}{|r|}{TP} & 64 & 1444 & 548 & 1214 & 1416 & 0 & 4686 \\ \hline
\multicolumn{1}{|r|}{7m} & 74 & 178 & 34 & 278 & 218 & 0 & 782 \\ \hline
\multicolumn{1}{|r|}{All} & 324 & 2066 & 672 & 2296 & 2420 & 42 & 7820 \\ \hline
\end{tabular}
\caption{Executives' observed time, broken down per array configuration type, for DSA run}
\label{table:dsa-hours-per-array}
\end{table}

\begin{table}[h!]
\centering
\begin{tabular}{c|c|c|c|c|c|c|c|} 
\cline{2-8}
 & \multicolumn{7}{c|}{Executive observation time $[h]$} \\ \hline
 \multicolumn{1}{|r|}{Array Type} & CL	& EA & EA\_NA &	EU & NA & OTHER & Total \\ \hline
 \multicolumn{1}{|r|}{12m} &	230 & 	472 &	110 &	778 & 	746 &	42 & 2378 \\ \hline
 \multicolumn{1}{|r|}{TP} & 116 & 944 & 632 &	1744 &	1298 & 	0 &	4734 \\ \hline
 \multicolumn{1}{|r|}{7m} & 74	& 178 & 34 & 278 & 216 & 0 & 780 \\ \hline
 \multicolumn{1}{|r|}{All} & 420 & 1594 & 776 & 2800 & 2260 & 42 & 7892 \\ \hline
\end{tabular}
\caption{Executives' observed time, broken down per array configuration type, for DRR run}
\label{table:drr-hours-per-array}
\end{table}

From these tables it is possible to see, that the DRR algorithm was able to, marginally, use better the array configurations available time. DRR was able to schedule 7892 $[h]$ in comparison with DSA, which scheduled 7820 $[h]$. The main contributor to this difference was the schedule for the $TP$ array, on which the DRR was able to allocate 48 $[h]$ more. 

Looking at table~\ref{table:requested-time-tp} it is possible to verify that, $TP$ Array Configuration is quite oversubscribed with requests and both algorithms were able to schedule almost the whole available time. This does this array configuration case interesting to further analyze in detail. 

On contrary the $7m$ array configuration, which is also available for the whole observation cycle, clearly lacks of observation requests (see table~\ref{table:requested-time-7m}). This is an extreme case that may be worth to analyze too, to see how algorithms behave in these situations.

In other hand the case of $12m$ configurations, it is possible to appreciate, in the details, cases where it looks like the array configuration was oversubscribed and as well cases where the array configuration was not much requests. The difference between the $12m$ configuration and the $7m$ and $TP$ is that they were available for a much shorter period within the observing season. Also further analysis of the most relevant $12m$ array configuration will be done.

From tables~\ref{table:dsa-hours-per-array} and~\ref{table:drr-hours-per-array}, it is also possible to infer that somehow the DRR algorithm affected to EA and NA Executives decreasing in around $22.85\%$ and $6.61\%$ the number of hours assigned to them, respectively. At the same time, DRR benefited to CL, EA\_NA and EU executives increasing in $22.86\%$, $15.48\%$ and $21.95\%$ the number of hours assigned to them, respectively.

Note that when EA\_NA executive is considered part of EA and NA then the number of hours for the EA and NA will be: $2402\,[h]$ and $2756\,[h]$, respectively for the DSA run; $1982\,[h]$ and $2648\,[h]$ respectively for DRR run. Leaving the decreasing of the time assigned to EA and NA to $17.48\%$ and $3.92\%$ respectively.  

A qualitative analysis of the time share given to each executive is presented in table~\ref{table:dsa-percentage-per-array} for the DSA, and the same analysis for the DRR is presented in table~\ref{table:drr-percentage-per-array}. 
\begin{table}[h!]
\centering
\begin{tabular}{c|c|c|c|c|c|c|} 
\cline{2-7}
 & \multicolumn{6}{c|}{Executive observation time $[\%]$} \\ \hline
\multicolumn{1}{|r|}{Array Type} & CL	& EA & EA\_NA &	EU & NA & OTHER \\ \hline
\multicolumn{1}{|r|}{12m} & 7.908 & 18.878 & 3.827 & 34.184 & 33.418 & 1.786 \\ \hline
\multicolumn{1}{|r|}{TP} & 1.366 & 30.815 & 11.694 & 25.907 & 30.218 & 0 \\ \hline
\multicolumn{1}{|r|}{7m} & 9.463 & 22.762 & 4.348 & 35.55 & 27.877 & 0 \\ \hline
\multicolumn{1}{|r|}{All} & 4.143 & 26.419 & 8.593 & 29.361 & 30.946 & 0.537 \\ \hline
\end{tabular}
\caption{Time share assigned to each executive, broken down per array configuration type, for DSA run}
\label{table:dsa-percentage-per-array}
\end{table}

\begin{table}[h!]
\centering
\begin{tabular}{c|c|c|c|c|c|c|} 
\cline{2-7}
 & \multicolumn{6}{c|}{Executive observation time $[\%]$} \\ \hline
\multicolumn{1}{|r|}{Array Type} & CL	& EA & EA\_NA &	EU & NA & OTHER \\ \hline
\multicolumn{1}{|r|}{12m} & 9.672 & 19.849 & 4.626 & 32.717 & 31.371 & 1.766 \\ \hline
\multicolumn{1}{|r|}{TP} & 2.45 & 19.941 & 13.35 & 36.84 & 27.419 & 0 \\ \hline
\multicolumn{1}{|r|}{7m} & 9.487 & 22.821 & 4.359 & 35.641 & 27.692 & 0 \\ \hline
\multicolumn{1}{|r|}{All} & 5.322 & 20.198 & 9.833 & 35.479 & 28.637 & 0.532 \\ \hline
\end{tabular}
\caption{Time share assigned to each executive, broken down per array configuration type, for DRR run}
\label{table:drr-percentage-per-array}
\end{table}

From these tables it is possible to see a better time sharing among the different executives when DRR is compared with DSA. Comparing with table~\ref{table:input-executive} it is possible to observe that, the time share values are closer with the share yielded by the DRR. Although there is a noticeable difference for NA executive, however adding its NA\_EA share, then NA share become $33.55\%$; the same can be done for EA executive, leaving its share at $25.11\%$.

As promised before, particular array configurations will be further analyzed to see how both algorithm behave in time periods of 1 week. The qualitative analysis consists on show how fair perform the particular algorithms within the week.

The qualitative analysis of C34-2 array configuration is presented in figure~\ref{fig:dsa-c34-2-exec} for the DSA instance and in figure~\ref{fig:drr-c34-2-exec} for DRR instance. Among all the $12m$ configurations, C34-2 was chosen because it looks like it have a normal request of observation time and the executives requested a well distributed amount of time. 

\begin{figure}[h!]
\centering
	\begin{subfigure}[b]{0.49\textwidth}
		\includegraphics[width=\textwidth]{images/results/dsa-c34-2_1a}
        \caption{Raw data} 
    \end{subfigure} 
    \begin{subfigure}[b]{0.49\textwidth}
    		\includegraphics[width=\textwidth]{images/results/dsa-c34-2_1b}
            \caption{Normalized time per week} 
    \end{subfigure}
    \caption{Observation time for C34-2 array, broken-down per executive, using DSA}
    \label{fig:dsa-c34-2-exec}
\end{figure}

\begin{figure}[h!]
\centering
	\begin{subfigure}[b]{0.49\textwidth}
		\includegraphics[width=\textwidth]{images/results/drr-c34-2_1a}
        \caption{Raw data} 
    \end{subfigure} 
    \begin{subfigure}[b]{0.49\textwidth}
    		\includegraphics[width=\textwidth]{images/results/drr-c34-2_1b}
            \caption{Normalized time per week} 
    \end{subfigure}
    \caption{Observation time for C34-2 array, broken-down per executive, using DRR}
    \label{fig:drr-c34-2-exec}
\end{figure}

Comparing both figures, it is possible to see at naked eye a much better distribution of the time among all the executives when DRR was used as scheduling algorithm. This behavior can also be seen in the last week (shown by the last column for the raw data charts), when the array used much less time for observations, given that there no more available scheduling blocks.

\begin{figure}[h!]
\centering
	\begin{subfigure}[b]{0.49\textwidth}
		\includegraphics[width=\textwidth]{images/results/dsa-tp_a}
        \caption{Raw data} 
    \end{subfigure} 
    \begin{subfigure}[b]{0.49\textwidth}
    		\includegraphics[width=\textwidth]{images/results/dsa-tp_b}
            \caption{Normalized time per week} 
    \end{subfigure}
    \caption{Observation time for $TP$ array, broken-down per executive, using DSA}
    \label{fig:dsa-tp-exec}
\end{figure}

\begin{figure}[h!]
\centering
	\begin{subfigure}[b]{0.49\textwidth}
		\includegraphics[width=\textwidth]{images/results/drr-tp_a}
        \caption{Raw data} 
    \end{subfigure} 
    \begin{subfigure}[b]{0.49\textwidth}
    		\includegraphics[width=\textwidth]{images/results/drr-tp_b}
            \caption{Normalized time per week} 
    \end{subfigure}
    \caption{Observation time for $TP$ array, broken-down per executive, using DRR}
    \label{fig:drr-tp-exec}
\end{figure}

\begin{figure}[h!]
\centering
	\begin{subfigure}[b]{0.49\textwidth}
		\includegraphics[width=\textwidth]{images/results/dsa-7m_a}
        \caption{Raw data} 
    \end{subfigure} 
    \begin{subfigure}[b]{0.49\textwidth}
    		\includegraphics[width=\textwidth]{images/results/dsa-7m_b}
            \caption{Normalized time per week} 
    \end{subfigure}
    \caption{Observation time for $7m$ array, broken-down per executive, using DSA}
    \label{fig:dsa-7m-exec}
\end{figure}

\begin{figure}[h!]
\centering
	\begin{subfigure}[b]{0.49\textwidth}
		\includegraphics[width=\textwidth]{images/results/drr-7m_a}
        \caption{Raw data} 
    \end{subfigure} 
    \begin{subfigure}[b]{0.49\textwidth}
    		\includegraphics[width=\textwidth]{images/results/drr-7m_b}
            \caption{Normalized time per week} 
    \end{subfigure}
    \caption{Observation time for $7m$ array, broken-down per executive, using DRR}
    \label{fig:drr-7m-exec}
\end{figure}

Until this point it is possible to acknowledge that DRR algorithm performs much better in terms of global fairness, giving to each participant executive a more equitable time share, in comparison to what has been proposed. Nevertheless this is not the only one parameter to consider the effectiveness of the algorithm, another very important parameter is how much projects the scheduling algorithm is able to complete for the given observing season.

The summary for DSA algorithm instance is presented in table~\ref{table:dsa-run-summary}. The summary for DRR instance is presented in table~\ref{table:drr-run-summary}.

\begin{table}[h!]
\centering
\begin{tabular}{|r|c|c|c|} \hline
 N Projects & A-graded & B-graded & C-graded \\ \hline
 Completed & 24 & 126 & 43 \\ \hline
 Incomplete & 7 & 118 & 76 \\ \hline
 Non-Started & 2 & 58 & 472 \\ \hline
\end{tabular}
\caption{Observation projects yield for DSA run}
\label{table:dsa-run-summary}
\end{table}

\begin{table}[h!]
\centering
\begin{tabular}{|r|c|c|c|} \hline
 N Projects & A-graded & B-graded & C-graded \\ \hline
 Completed & 20 & 125 & 47 \\ \hline
 Incomplete & 11 & 123 & 81 \\ \hline
 Non-Started & 2 & 55 & 472 \\ \hline
\end{tabular}
\caption{Observation projects yield for DRR run}
\label{table:drr-run-summary}
\end{table}

From the tables it is possible to see that DSA performs better completing projects, specially A-graded projects. This can be explained because DSA can be modeled as a DRR with only one service flow, therefore the better graded Scheduling Blocks will be always on top of the other SBs, having a chance to complete more better graded projects. However, the DRR is able to start more projects, but as well leaves more of these started projects incomplete. Having that in mind a special scorer was introduced to pump-up the score of the started projects. But at the end, the result did not vary significantly from what is presented in the table~\ref{table:drr-run-summary}.

\section{Testing solution for the Array configuration planning problem} 
As explained in section~\ref{sec:array-config-plan}, the algorithm will, first, narrow the search space based on the A-graded Scheduling blocks, the visibility for them is presented in figure~\ref{fig:results-sb-critical-set}. At same time each SB will be assigned to one or more Array Configurations. For the case of the input data used in this work, each A-graded SB is mapped to just one Array Configuration.

\begin{figure}
        \centering
        \begin{subfigure}[b]{0.45\textwidth}
                \includegraphics[width=\textwidth]{images/c34-1_sources}
                \caption{C34-1} 
        \end{subfigure} 
        ~ %
%
        \begin{subfigure}[b]{0.45\textwidth}
                \includegraphics[width=\textwidth]{images/c34-2_sources}
                \caption{C34-2}
        \end{subfigure}

        \begin{subfigure}[b]{0.45\textwidth}
                \includegraphics[width=\textwidth]{images/c34-3_sources}
                \caption{C34-3}
        \end{subfigure}
        ~ 
        \begin{subfigure}[b]{0.45\textwidth}
                \includegraphics[width=\textwidth]{images/c34-4_sources}
                \caption{C34-4}
        \end{subfigure}% 
        
        \begin{subfigure}[b]{0.45\textwidth}
                \includegraphics[width=\textwidth]{images/c34-5_sources}
                \caption{C34-5}
        \end{subfigure}
        ~
        \begin{subfigure}[b]{0.45\textwidth}
                \includegraphics[width=\textwidth]{images/c34-6_sources}
                \caption{C34-6}
        \end{subfigure}
        
        \begin{subfigure}[b]{0.45\textwidth}
                \includegraphics[width=\textwidth]{images/c34-7_sources}
                \caption{C34-7}
        \end{subfigure}           
        \caption{Visibility of A-graded Scheduling Blocks for $12\,[m]$ Array Configurations}
		\label{fig:results-sb-critical-set}
\end{figure}

Then the LST intervals proposed by the Scheduling Block classification software can be seen in table~\ref{table:lst-int-prop}.

\begin{table}[h!]
\begin{center}
\begin{tabular}{|c|c|c|}
\hline
Array configuration & Start time (LST) & End time (LST) \\ \hline
C34-1 & 5.61692 & 15.45551 \\ \hline
C34-1 & 13.30675 & 23.14534 \\ \hline
C34-2 & 9.86519 & 19.703777 \\ \hline
C34-2 & 13.22811 & 23.06670 \\ \hline
C34-3 & 4.44851 & 14.28710 \\ \hline
C34-3 & 9.76941 & 19.60800 \\ \hline
C34-3 & 12.97411 & 22.81270 \\ \hline
C34-4 & 5.61692 & 15.45551 \\ \hline
C34-5 & 9.93925 & 19.77784 \\ \hline
C34-6 & 21.92851 & 7.76710 \\ \hline
C34-7 & 2.77808 & 12.61667 \\ \hline
C34-7 & 4.94547 & 14.78406 \\ \hline
C34-7 & 9.76941 & 19.60800 \\ \hline
\end{tabular}
\end{center}
\caption[LST intervals proposed by the Scheduling Blocks categorization]
{LST intervals proposed by the Scheduling blocks categorization described in section~\ref{sec:array-sb-classification}}
\label{table:lst-int-prop}
\end{table}