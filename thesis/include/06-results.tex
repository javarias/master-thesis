\newpage
\section{Tests and results}

Two sets of tests will be performed to benchmark the algorithms proposed for both problems presented in this work.
The first set will show how the solution given to ``Astronomical observation scheduling problem'', focusing the comparison with current DSA implementation done for ALMA.
The second set will show the feasibility of the proposed algorithm to solve the ``Array configuration planning problem'', focusing in the results for different scenarios.

All the tests will be run on a 2 x Intel Xeon X5675 running at 3.06 $[GHz]$, with 48 $[GiB]$ of RAM DDR3 1333 $[MHz]$, configured in 6 channels (2 x QPI 6.4 $[GT/s]$). Running Red Hat Enterprise Linux 5.9 (x86 PAE) and Oracle's Java (JRE) 1.7.0\_51.

\subsection{Testing the algorithms for the Astronomical observation scheduling problem}

\subsubsection{Setup and objectives}
The main feature of the DRR algorithm, presented in this work, is to try to keep the observation time given to each Executive, according to the values presented in table~\ref{table:input-executive}, along the time array configuration life. For this reason, a comparison between the ALMA's DSA and the DRR will be done. 

\begin{table}[b]
\begin{center}
\begin{tabular}{|c|c|c|}
\hline
Array configuration & Start date (UTC) & End date (UTC) \\ \hline
C34-6 & 2014-06-01 & 2014-07-14 \\ \hline
C34-7 & 2014-07-15 & 2014-08-31 \\ \hline
C34-6 & 2014-12-01 & 2014-12-31 \\ \hline
C34-5 & 2015-01-01 & 2015-01-31 \\ \hline
C34-4 & 2015-03-01 & 2015-03-31 \\ \hline
C34-3 & 2015-04-01 & 2015-04-30 \\ \hline
C34-2 & 2015-05-01 & 2015-05-31 \\ \hline
C34-1 & 2015-06-01 & 2015-06-30 \\ \hline
C34-4 & 2015-07-01 & 2015-07-31 \\ \hline
C34-6 & 2015-08-01 & 2015-08-30 \\ \hline
C34-7 & 2015-09-01 & 2015-10-01 \\ \hline
7m & 2014-06-01 & 2015-10-01 \\ \hline
TP & 2014-06-01 & 2015-10-01 \\ \hline
\end{tabular}
\end{center}
\caption{Array configuration plan used to compare DRR algorithm vs DSA algorithm}
\label{table:drr-test-array-config}
\end{table}

The metric to compare both algorithm will be the percentage of time given to each executive every week, every two weeks, every four weeks and for the whole array configuration duration. At the same time, it will be analyzed the impact of the DRR algorithm for the project completion rate at the end of the whole simulation, checking if there is any adverse effect over this parameter. The completion rate will be based at project level, accounting projects completed, started but non-completed and non-started, these categories will be broken down based on the projects science grade given: $A$, $B$ and $C$.

\subsubsection{Fairness comparison analysis}
The test to benchmark both algorithms will be based on the input data described in section~\ref{sec:input-data}. Although the array configuration planning will be fixed to the one presented in table~\ref{table:drr-test-array-config}

The total amount of hours that the algorithm were able to schedule is summarized in table~\ref{table:dsa-hours-per-array} for the DSA execution, and in table~\ref{table:drr-hours-per-array}. Every DSA and DRR execution takes little less than $1\,[min]$

\begin{table}[t]
\centering
\begin{tabular}{c|c|c|c|c|c|c|c|} 
\cline{2-8}
 & \multicolumn{7}{c|}{Executive observation time $[h]$} \\ \hline
\multicolumn{1}{|r|}{Array Type} & CL	& EA & EA\_NA &	EU & NA & OTHER & Total \\ \hline
\multicolumn{1}{|r|}{12m} & 186 & 444 & 90 & 804 & 786 & 42 & 2352 \\ \hline
\multicolumn{1}{|r|}{TP} & 64 & 1444 & 548 & 1214 & 1416 & 0 & 4686 \\ \hline
\multicolumn{1}{|r|}{7m} & 74 & 178 & 34 & 278 & 218 & 0 & 782 \\ \hline
\multicolumn{1}{|r|}{All} & 324 & 2066 & 672 & 2296 & 2420 & 42 & 7820 \\ \hline
\end{tabular}
\caption{Executives' observed time, broken down per array configuration type, for DSA run}
\label{table:dsa-hours-per-array}
\end{table}

\begin{table}[t]
\centering
\begin{tabular}{c|c|c|c|c|c|c|c|} 
\cline{2-8}
 & \multicolumn{7}{c|}{Executive observation time $[h]$} \\ \hline
 \multicolumn{1}{|r|}{Array Type} & CL	& EA & EA\_NA &	EU & NA & OTHER & Total \\ \hline
 \multicolumn{1}{|r|}{12m} &	230 & 	472 &	110 &	778 & 	746 &	42 & 2378 \\ \hline
 \multicolumn{1}{|r|}{TP} & 116 & 944 & 632 &	1744 &	1298 & 	0 &	4734 \\ \hline
 \multicolumn{1}{|r|}{7m} & 74	& 178 & 34 & 278 & 216 & 0 & 780 \\ \hline
 \multicolumn{1}{|r|}{All} & 420 & 1594 & 776 & 2800 & 2260 & 42 & 7892 \\ \hline
\end{tabular}
\caption{Executives' observed time, broken down per array configuration type, for DRR run}
\label{table:drr-hours-per-array}
\end{table}

From these tables it is possible to see, that the DRR algorithm was able to, marginally, use better the array configurations available time. DRR was able to schedule 7892 $[h]$ in comparison with DSA, which scheduled 7820 $[h]$. The main contributor to this difference was the schedule for the $TP$ array, on which the DRR was able to allocate 48 $[h]$ more. 

Looking at table~\ref{table:requested-time-tp} it is possible to verify that, $TP$ Array Configuration is quite oversubscribed with requests and both algorithms were able to schedule almost the whole available time. This does this array configuration case interesting to further analyze in detail. 

On contrary the $7m$ array configuration, which is also available for the whole observation cycle, clearly lacks of observation requests (see table~\ref{table:requested-time-7m}). This is an extreme case that may be worth to analyze too, to see how algorithms behave in these situations.

In other hand the case of $12m$ configurations, it is possible to appreciate, in the details, cases where it looks like the array configuration was oversubscribed and as well cases where the array configuration was not much requests. The difference between the $12m$ configuration and the $7m$ and $TP$ is that they were available for a much shorter period within the observing season. Also further analysis of the most relevant $12m$ array configuration will be done.

From tables~\ref{table:dsa-hours-per-array} and~\ref{table:drr-hours-per-array}, it is also possible to infer that somehow the DRR algorithm affected to EA and NA Executives decreasing in around $22.85\%$ and $6.61\%$ the number of hours assigned to them, respectively. At the same time, DRR benefited to CL, EA\_NA and EU executives increasing in $22.86\%$, $15.48\%$ and $21.95\%$ the number of hours assigned to them, respectively.

Note that when EA\_NA executive is considered part of EA and NA then the number of hours for the EA and NA will be: $2402\,[h]$ and $2756\,[h]$, respectively for the DSA run; $1982\,[h]$ and $2648\,[h]$ respectively for DRR run. Leaving the decreasing of the time assigned to EA and NA to $17.48\%$ and $3.92\%$ respectively.  

\begin{table}[t]
\centering
\begin{tabular}{c|c|c|c|c|c|c|} 
\cline{2-7}
 & \multicolumn{6}{c|}{Executive observation time $[\%]$} \\ \hline
\multicolumn{1}{|r|}{Array Type} & CL	& EA & EA\_NA &	EU & NA & OTHER \\ \hline
\multicolumn{1}{|r|}{12m} & 7.908 & 18.878 & 3.827 & 34.184 & 33.418 & 1.786 \\ \hline
\multicolumn{1}{|r|}{TP} & 1.366 & 30.815 & 11.694 & 25.907 & 30.218 & 0 \\ \hline
\multicolumn{1}{|r|}{7m} & 9.463 & 22.762 & 4.348 & 35.55 & 27.877 & 0 \\ \hline
\multicolumn{1}{|r|}{All} & 4.143 & 26.419 & 8.593 & 29.361 & 30.946 & 0.537 \\ \hline
\end{tabular}
\caption{Time share assigned to each executive, broken down per array configuration type, for DSA run}
\label{table:dsa-percentage-per-array}
\end{table}

\begin{table}[t]
\centering
\begin{tabular}{c|c|c|c|c|c|c|} 
\cline{2-7}
 & \multicolumn{6}{c|}{Executive observation time $[\%]$} \\ \hline
\multicolumn{1}{|r|}{Array Type} & CL	& EA & EA\_NA &	EU & NA & OTHER \\ \hline
\multicolumn{1}{|r|}{12m} & 9.672 & 19.849 & 4.626 & 32.717 & 31.371 & 1.766 \\ \hline
\multicolumn{1}{|r|}{TP} & 2.45 & 19.941 & 13.35 & 36.84 & 27.419 & 0 \\ \hline
\multicolumn{1}{|r|}{7m} & 9.487 & 22.821 & 4.359 & 35.641 & 27.692 & 0 \\ \hline
\multicolumn{1}{|r|}{All} & 5.322 & 20.198 & 9.833 & 35.479 & 28.637 & 0.532 \\ \hline
\end{tabular}
\caption{Time share assigned to each executive, broken down per array configuration type, for DRR run}
\label{table:drr-percentage-per-array}
\end{table}

A qualitative analysis of the time share given to each executive is presented in table~\ref{table:dsa-percentage-per-array} for the DSA, and the same analysis for the DRR is presented in table~\ref{table:drr-percentage-per-array}. 

From these tables it is possible to see a better time sharing among the different executives when DRR is compared with DSA. Comparing with table~\ref{table:input-executive} it is possible to observe that, the time share values are closer with the share yielded by the DRR. Although there is a noticeable difference for NA executive, however adding its NA\_EA share, then NA share become $33.55\%$; the same can be done for EA executive, leaving its share at $25.11\%$.

\subsubsection{Detailed fairness analysis}
Particular array configurations will be further analyzed to see how both algorithm behave in time periods of 1 week. The qualitative analysis consists on show how fair perform the particular algorithms within the week for each executive.

\begin{figure}[t]
\centering
	\begin{subfigure}[b]{0.49\textwidth}
		\includegraphics[width=\textwidth]{images/results/dsa-c34-2_1a}
        \caption{Raw data} 
    \end{subfigure} 
    \begin{subfigure}[b]{0.49\textwidth}
    		\includegraphics[width=\textwidth]{images/results/dsa-c34-2_1b}
            \caption{Normalized time per week} 
    \end{subfigure}
    \caption{Observation time for C34-2 array, broken-down per executive, using DSA}
    \label{fig:dsa-c34-2-exec}
\end{figure}

\begin{figure}[t]
\centering
	\begin{subfigure}[b]{0.49\textwidth}
		\includegraphics[width=\textwidth]{images/results/drr-c34-2_1a}
        \caption{Raw data} 
    \end{subfigure} 
    \begin{subfigure}[b]{0.49\textwidth}
    		\includegraphics[width=\textwidth]{images/results/drr-c34-2_1b}
            \caption{Normalized time per week} 
    \end{subfigure}
    \caption{Observation time for C34-2 array, broken-down per executive, using DRR}
    \label{fig:drr-c34-2-exec}
\end{figure}

The qualitative analysis of C34-2 array configuration is presented in figure~\ref{fig:dsa-c34-2-exec} for the DSA instance and in figure~\ref{fig:drr-c34-2-exec} for DRR instance. Among all the $12m$ configurations, C34-2 was chosen because it looks like it have a normal request of observation time and the executives requested a well distributed amount of time. 

Comparing both figures, it is possible to see at naked eye a much better distribution of the time among all the executives when DRR was used as scheduling algorithm. This behavior can also be seen in the last week (shown by the last column for the raw data charts), when the array used much less time for observations, given that there no more available scheduling blocks.

\begin{figure}[t]
\centering
	\begin{subfigure}[b]{0.49\textwidth}
		\includegraphics[width=\textwidth]{images/results/dsa-tp_a}
        \caption{Raw data} 
    \end{subfigure} 
    \begin{subfigure}[b]{0.49\textwidth}
    		\includegraphics[width=\textwidth]{images/results/dsa-tp_b}
            \caption{Normalized time per week} 
    \end{subfigure}
    \caption{Observation time for $TP$ array, broken-down per executive, using DSA}
    \label{fig:dsa-tp-exec}
\end{figure}

\begin{figure}[t]
\centering
	\begin{subfigure}[b]{0.49\textwidth}
		\includegraphics[width=\textwidth]{images/results/drr-tp_a}
        \caption{Raw data} 
    \end{subfigure} 
    \begin{subfigure}[b]{0.49\textwidth}
    		\includegraphics[width=\textwidth]{images/results/drr-tp_b}
            \caption{Normalized time per week} 
    \end{subfigure}
    \caption{Observation time for $TP$ array, broken-down per executive, using DRR}
    \label{fig:drr-tp-exec}
\end{figure}

The qualitative analysis for $TP$ array configuration is presented in figure~\ref{fig:dsa-tp-exec} for the DSA instance. Figure~\ref{fig:drr-tp-exec} introduced the qualitative analysis for DRR. This is the best array configuration to compare DRR with DSA in weekly basis because of the over-subscription of Scheduling Blocks, which is well-balanced among the executives.

Since most of the available time for observation is used by both algorithm, it is possible to see that the raw data and the normalized charts are almost the same, it only differs near the end of the observing season, where less hours are scheduled mostly due completion of the Observation projects and Scheduling Blocks.

Comparing the fairness between DSA and DRR, here it is clearly that DRR is fairer than DSA, this is quite evident specially at the beginning of the observation season (first half columns in the charts). When DRR ran, there is a spike for CL executive which is caused by the unfairness accumulation before its spike. In the second half of the DRR chart the distribution is more biased to EU executive, which is opposite to DSA where at the end of the observing season the share is more biased to NA.

\begin{figure}[t]
\centering
	\begin{subfigure}[b]{0.49\textwidth}
		\includegraphics[width=\textwidth]{images/results/dsa-7m_a}
        \caption{Raw data} 
    \end{subfigure} 
    \begin{subfigure}[b]{0.49\textwidth}
    		\includegraphics[width=\textwidth]{images/results/dsa-7m_b}
            \caption{Normalized time per week} 
    \end{subfigure}
    \caption{Observation time for $7m$ array, broken-down per executive, using DSA}
    \label{fig:dsa-7m-exec}
\end{figure}

\begin{figure}[t]
\centering
	\begin{subfigure}[b]{0.49\textwidth}
		\includegraphics[width=\textwidth]{images/results/drr-7m_a}
        \caption{Raw data} 
    \end{subfigure} 
    \begin{subfigure}[b]{0.49\textwidth}
    		\includegraphics[width=\textwidth]{images/results/drr-7m_b}
            \caption{Normalized time per week} 
    \end{subfigure}
    \caption{Observation time for $7m$ array, broken-down per executive, using DRR}
    \label{fig:drr-7m-exec}
\end{figure}

The qualitative analysis for $7m$ array configuration is presented in figure~\ref{fig:dsa-7m-exec} for the DSA instance. Figure~\ref{fig:drr-7m-exec} introduced the qualitative analysis for DRR. This particular array configuration is interesting because it neither is oversubscribed nor well balanced. As can be predicted this array configuration used less than one-third of the available time, the charts only show the total time when it is non-zero.

As expected there is no surprises, looks like DRR balance better the time sharing among all the executives, this is clear at the beginning of the season. 

\subsubsection{Project completion analysis}

Until this point it is possible to acknowledge that DRR algorithm performs much better in terms of global fairness, giving to each participant executive a more equitable time share, in comparison to what has been proposed. Nevertheless this is not the only one parameter to consider the effectiveness of the algorithm, another very important parameter is how much projects the scheduling algorithm is able to complete for the given observing season.

The summary for DSA algorithm instance is presented in table~\ref{table:dsa-run-summary}. The summary for DRR instance is presented in table~\ref{table:drr-run-summary}.

From the tables it is possible to see that DSA performs better completing projects, specially A-graded projects. This can be explained because DSA can be modeled as a DRR with only one service flow, therefore the better graded Scheduling Blocks will be always on top of the other SBs, having a chance to complete more better graded projects. However, the DRR is able to start more projects, but as well leaves more of these started projects incomplete. Having that in mind a special scorer was introduced to pump-up the score of the started projects. But at the end, the result did not vary significantly from what is presented in the table~\ref{table:drr-run-summary}.

\begin{table}[t!]
\centering
\begin{tabular}{|r|c|c|c|} \hline
 N Projects & A-graded & B-graded & C-graded \\ \hline
 Completed & 24 & 126 & 43 \\ \hline
 Incomplete & 7 & 118 & 76 \\ \hline
 Non-Started & 2 & 58 & 472 \\ \hline
\end{tabular}
\caption{Observation projects yield for DSA run}
\label{table:dsa-run-summary}
\end{table}

\begin{table}[t!]
\centering
\begin{tabular}{|r|c|c|c|} \hline
 N Projects & A-graded & B-graded & C-graded \\ \hline
 Completed & 20 & 125 & 47 \\ \hline
 Incomplete & 11 & 123 & 81 \\ \hline
 Non-Started & 2 & 55 & 472 \\ \hline
\end{tabular}
\caption{Observation projects yield for DRR run}
\label{table:drr-run-summary}
\end{table}

\subsection{Testing solution for the Array configuration planning problem}
This tests will show whether the procedure described in chapter~\ref{sec:array-config-plan} is able to produce valid solutions for the problem or not. Also different instances of the problem will be considered to show how the algorithms behave related to when difficulty of the problem increases. Each problem instance will be defined as as the number of A-graded projects that the algorithm shall complete. The idea behind this is to know how many iterations, in average (due the stochastic nature of the algorithm design), require to get a feasible result for that problem instance.

For all the tests done, the DRR algorithm is the one used as the scheduling algorithm to solve the Astronomical observation scheduling problem. The same procedure can be applied using the DSA. The main difference is that using the DSA would solve higher problem instances, due the DSA tends to complete more A-graded projects, however it is no expected a difference in the behavior for difficult problems, when the solution proposed is used to solve the Array configuration planning problem.   

As explained in section~\ref{sec:acpp-sa-alg-design}, the algorithm will first, narrow the search space based on the A-graded Scheduling blocks.  The visibility for these Scheduling blocks is presented in figure~\ref{fig:results-sb-critical-set}. Once the visibility of the sources, then each SB will be assigned to one or more Array Configurations. For the case of the input data used in this work, each A-graded SB sky source is mapped to just one Array Configuration.

\begin{figure}[htbp]
        \centering
        \begin{subfigure}[b]{0.45\textwidth}
                \includegraphics[width=\textwidth]{images/c34-1_sources}
                \caption{C34-1} 
        \end{subfigure} 
        ~ %
%
        \begin{subfigure}[b]{0.45\textwidth}
                \includegraphics[width=\textwidth]{images/c34-2_sources}
                \caption{C34-2}
        \end{subfigure}

        \begin{subfigure}[b]{0.45\textwidth}
                \includegraphics[width=\textwidth]{images/c34-3_sources}
                \caption{C34-3}
        \end{subfigure}
        ~ 
        \begin{subfigure}[b]{0.45\textwidth}
                \includegraphics[width=\textwidth]{images/c34-4_sources}
                \caption{C34-4}
        \end{subfigure}% 
        
        \begin{subfigure}[b]{0.45\textwidth}
                \includegraphics[width=\textwidth]{images/c34-5_sources}
                \caption{C34-5}
        \end{subfigure}
        ~
        \begin{subfigure}[b]{0.45\textwidth}
                \includegraphics[width=\textwidth]{images/c34-6_sources}
                \caption{C34-6}
        \end{subfigure}
        
        \begin{subfigure}[b]{0.45\textwidth}
                \includegraphics[width=\textwidth]{images/c34-7_sources}
                \caption{C34-7}
        \end{subfigure}           
        \caption{Visibility of A-graded Scheduling Blocks for $12\,[m]$ Array Configurations}
		\label{fig:results-sb-critical-set}
\end{figure}

Then the LST intervals proposed by the Scheduling Block classification software can be seen in table~\ref{table:lst-int-prop}.

\begin{table}[htbp]
\begin{center}
\begin{tabular}{|c|c|c|}
\hline
Array configuration & Start time (LST) & End time (LST) \\ \hline
C34-1 & 5.61692 & 15.45551 \\ \hline
C34-1 & 13.30675 & 23.14534 \\ \hline
C34-2 & 9.86519 & 19.703777 \\ \hline
C34-2 & 13.22811 & 23.06670 \\ \hline
C34-3 & 4.44851 & 14.28710 \\ \hline
C34-3 & 9.76941 & 19.60800 \\ \hline
C34-3 & 12.97411 & 22.81270 \\ \hline
C34-4 & 5.61692 & 15.45551 \\ \hline
C34-5 & 9.93925 & 19.77784 \\ \hline
C34-6 & 21.92851 & 7.76710 \\ \hline
C34-7 & 2.77808 & 12.61667 \\ \hline
C34-7 & 4.94547 & 14.78406 \\ \hline
C34-7 & 9.76941 & 19.60800 \\ \hline
\end{tabular}
\end{center}
\caption[LST intervals proposed by the Scheduling Blocks categorization]
{LST intervals proposed by the Scheduling blocks categorization described in section~\ref{sec:array-sb-classification}}
\label{table:lst-int-prop}
\end{table}

Comparing both, the sources elevation vs LST intervals proposed by the algorithm, it is possible to see that most of A-graded SB's sky sources are covered on those proposed LST intervals.

\begin{figure}[htbp]
\begin{center}
\includegraphics[width=\textwidth]{images/results/acpp-instances}
\end{center}
\caption{Simulated Annealing algorithm behavior related to the problem difficulty increasing}
\label{fig:acpp-behavior}
\end{figure}

Different instances of the problem were tested several times, the summary of the testing can be seen in the graph represented by figure~\ref{fig:acpp-behavior}. A maximum of 1000 iterations are allowed, if a solution is not found within that number of iterations, then the problem will be considered non-solved. A full execution of 1000 iterations takes around $800\,[min]$.


In the graph, the x axis represents the problem instance to solve, the y axis represents the number of iterations that SA algorithm used, the dots represent the average number of iterations that the SA algorithm requires to find a valid solution for the problem instance and the vertical bars are the error presented as the standard deviation. 

Here is possible the see that, for the instances before 20, the algorithms proposed easily finds a valid solution for the problem, requiring only 1 iteration to find a valid solution. From instance 20 there is an increment on the number of iterations required to find a solution, this increasing in the difficulty looks like exponential until instance 25, which is the most difficult problem instance that the algorithm is able to solve. Even though problem 25 was solved it was not solved always, this is represented by the error bar, which is bigger that the maximum allowed iterations (1000).

\begin{table}[htbp]
\begin{center}
\begin{tabular}{|c|c|c|}
\hline
Array configuration & Start date (UTC) & End date (UTC) \\ \hline
C34-2 & 2014-06-01 & 2014-06-29  \\ \hline
C34-3 & 2014-06-29 & 2014-07-27 \\ \hline
C34-1 & 2014-07-27 & 2014-08-24 \\ \hline
C34-6 & 2014-11-16 & 2014-12-14 \\ \hline
C34-7 & 2014-12-21 & 2015-01-18 \\ \hline
C34-3 & 2015-01-18 & 2015-02-15 \\ \hline
C34-4 & 2015-02-15 & 2015-03-15 \\ \hline
C34-7 & 2015-03-15 & 2015-04-12 \\ \hline
C34-1 & 2015-04-12 & 2015-05-10 \\ \hline
C34-7 & 2015-05-10 & 2015-06-07 \\ \hline
C34-5 & 2015-06-07 & 2015-07-05  \\ \hline
C34-2 & 2015-07-05 & 2015-08-02  \\ \hline
7m & 2014-06-01 & 2015-10-01 \\ \hline
TP & 2014-06-01 & 2015-10-01 \\ \hline
\end{tabular}
\end{center}
\caption{Best array configuration found by the SA algorithm}
\label{table:best-arr-conf}
\end{table}

Until now the best array configuration found by the SA algorithm is detailed in table~\ref{table:best-arr-conf}. Which it yields the project completion presented in table~\ref{table:best-solution-found} when the DRR algorithm is used.

\begin{table}[htbp]
\centering
\begin{tabular}{|r|c|c|c|} \hline
 N Projects & A-graded & B-graded & C-graded \\ \hline
 Completed & 25 & 148 & 66 \\ \hline
 Incomplete & 6 & 116 & 61 \\ \hline
 Non-Started & 2 & 38 & 31 \\ \hline
\end{tabular}
\caption{Observation projects completion yielded by the best array configuration found}
\label{table:best-solution-found}
\end{table}

\chapter{Conclusions}
This document has presented the ALMA's scheduling problem, analyzing in details the particularities found during the study of this problem. This problem has been separated in two different subproblems: the ``Astronomical observation scheduling problem'' describing the short and medium term scheduling subproblems proposed by ALMA; and the ``Array configuration planning problem'' detailing the long-term scheduling subproblem proposed by ALMA.

Solutions were discussed, implemented and validated for both problems presented in this work, although some simplifications were done during the discussion of these problems, all (or most of) the constraints are currently used in ALMA scheduling subsystem. Therefore, given the final results, the algorithms discussed in this work, can be used directly in ALMA observatory, however small modification are required in the current ALMA scheduling software, in particular for the solution for the Astronomical observation scheduling problem, where DRR needs to save the round robin pointers and deficit values.

The objectives (see section~\ref{sec:objectives}) were fulfilled as follows:
\begin{enumerate}
\item \textbf{To develop a new scheduling algorithm that meets the ALMA’s scientific, instrument and
weather constraints, and requirements; and successfully integrate it within ALMA software.
The development will be focused in try to solve the long-term scheduling problem, where
the algorithm shall provide a schedule for different array configurations within an
observing season.} The whole problem was introduced and discussed considering the constraints and requirements currently in use by the ALMA scheduling subsystem, the mathematical models and the algorithms proposed as solutions covers the short, medium and long term ALMA scheduling problem. A new approach is considered for the ``Astronomical observation scheduling problem'' a solution modeled after a fair scheduling algorithm, a modified deficit Round-Robin. But also, this work introduces the analysis and solution for the ``Array configuration planning problem'', which is the long-term problem proposed by ALMA.
\item \textbf{Define metrics to evaluate the algorithm performance in terms of idle-time or instrument
usage, scientific throughput and completed projects.} To validate the specific problem presented in ALMA, new metrics were introduced based on participant executive fairness within a fixed period of time, this work also considers previous defined metrics, such as number of projects completed, but including ALMA specific details, such as the scientific grade given to the projects and SBs. The software developed is based on the current simulator tools used in ALMA, therefore the adaptation to make the new algorithms developed in this work available for ALMA software is easy. 
\end{enumerate}

Therefore, the initial hypothesis was successfully proven, as \textit{it is possible to design and implement an algorithm for the ALMA scheduling problem, considering most of its particular constraints}.

The proposed model, algorithms and its implementations have the following features, for the ``Astronomical observation scheduling problem'':
\begin{itemize}
\item The time on what the model has been designed and the algorithm operates, has not been discretized as previous works.

\item The model considers dynamic constraints at job level (Scheduling Block) and as well in constraint level. Also introduces a very particular constraint, related to the time-share partition given to each executive participant of the telescope.

\item This work introduces a solution based on fair scheduling approach, proven that is feasible to adapt that kind of problem to the Astronomical observation scheduling problem.

\item The algorithm and the software can be easily adapted, making the solution usable on the ALMA's on-line system. In this environment the algorithm must be able to read real world data, changing constantly over the time. 

\item Although the model and solutions are particular to the ALMA scheduling problem, they can be adapted in more generic way and to be adapted to similar observatory problems.

\end{itemize} 

The proposed model, algorithms and its implementations have the following features, for the ``Array configuration planning problem'':
\begin{itemize}
\item The observing cycle has been discretized in terms of week, each array configuration must have a minimum lifespan of 4 weeks.

\item The approach used to solve the problem considers that the array configurations are input data, as well the projects and their initial scientific appraisal (grades and scores).

\item The long-term problem and solution presented deals with the current problem in commissioning stage of ALMA observatory, where the telescope is not operating 24/7.

\item The solution is only usable in simulation mode, this is an inherent characteristic of the long-term scheduling problem.

\item The model proposed is much more static in nature than the Astronomical observation scheduling problem, however, given the dependency on this last problem, all the characteristics are inherited and expanded in the Array configuration planning problem. Even though, is possible to have accurate simulation of weather patterns within the simulation, the problem may not be solved without the dependency on the Astronomical observation scheduling problem, given that it is very important to know how the sort-term solution behaves in medium and long-term simulations.

\end{itemize}

\section{Future work}
The solution given to ``Astronomical observation scheduling problem'' does not try to optimize for periods of time longer than the next period to observe, this is mostly because of the weather constraints that cannot be exactly foreseen. However, having a good weather forecast it may possible to verify whether it is feasible to implement an algorithm than can schedule the observation for longer period, like one day or at most two nights.

The long-term scheduling problem can be classified as a chicken or the egg dilemma, this is because there are two approaches to define it:
\begin{itemize}
\item The proposals submitted by the astronomers are based on the telescope capabilities offered for the observing cycle; or
\item The telescope capabilities will be based on what proposals the astronomers have been submitted for the observing cycle.
\end{itemize}

The current approach used in the telescope commissioning stage, is the first. Hence, the problem in this work has been defined accordingly. However, when ALMA observatory reaches full operations and capabilities, it could be that the second approach would be taken. Then, given that, the problem has to be completely redefined, and new a solution must be provided. Therefore, the ALMA scheduling problem, as presented in this document, might not be completely defined yet.

Perhaps another interesting approach to the classification of a critical SB set for an array configuration, is to try to classify the SBs based on weather patterns along the whole observing cycle. So far, it is known that there are certain months in the calendar year that are very bad or quite good, in terms of weather, to carry-on observation, an algorithm might introduce this ``intelligence'' to come with a better array configuration plan, or to come up with a better solution faster than the algorithms presented in this work.