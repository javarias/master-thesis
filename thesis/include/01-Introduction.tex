\chapter{Introduction}

State-of-the art astronomical facilities are costly to build and operate, so it is essential that these facilities 
must be operated as efficiently as possible, trying to maximize the scientific output. In modern astronomy has been
demonstrated the importance to use automatic modes, in the day-to-day operations.
Over the latest decades the problem has attracted more research, due new facilities have been
demonstrated that is unfeasible to try to schedule observations manually, given the complexity to satisfy
the astronomical and instrumental constraints imposed by the problem, and the number of scientific
proposals to be reviewed. In addition the dynamic nature of some constraints (i.e. the weather and
the uncertainty of completing an observation) make this problem even more difficult.

The Atacama Large Millimeter/submillimeter Array (ALMA) is a major collaboration effort between
European, North American and East Asian countries, under construction on the Chilean Chajnantor
plateau, at 5.000 meters of altitude. Currently it is the largest radio-telescope on Earth, when completed it
will have 66 antennas of 12 and 7 meters diameter, distributed over a wide extension, with up to 16 kilometers
of baseline separation. The ALMA interferometer will provide the possibility to be used as a single
array, or as up to six minor independent arrays or groups of antennas. Over the observing season
multiples arrays configurations will be available changing the antennas position from one place to
another. These places are already defined positions called pads, during normal operations at least two
arrays will be available each of one attached to different correlators computers. Since ALMA does not
observe in the visible spectrum, observations are not limited to night time only, thus a 24/7 operation
with as small downtime as possible is expected. However during commissioning operations, ALMA will operate on tied schedules having less than the half of the day-time to conduct observations.

Two different sub-problems have been identified: 
\begin{enumerate}
\item \textit{``Astronomical observation scheduling problem''}  \hfill \\
consists on determine a schedule for a given set of scheduling blocks, considering an array already created in a given fixed
period of time. This aims to complete most of the observation projects with highest priority. This is single-array problem instance;

\item \textit{``Array configuration planning problem''} \hfill \\
is on top of the Astronomical observation scheduling problem. It aims to determine a plan for the array configurations given as input,
for the observing season and for a given set of the scheduling blocks. This is the multiple-array problem instance.
\end{enumerate}

\section{Hypothesis}
The existing astronomical observation scheduling bibliography considers only the single-array (or telescope) problem instances,
treating the problem as a constraint satisfaction problem instead of an optimization problem.
ALMA scheduling software has an algorithm, the Dynamic Scheduling Algorithm
which does not try optimize the set of selected scheduling units
as a function of time. This algorithm suffers of
serious problems of executive time assignment, letting projects 
with the same or higher priority, but belonging to a different executive, to starve (never get observed).

Therefore the hypothesis presented is the following:
\textbf{``It is feasible to design and implement an algorithm for the ALMA scheduling problem 
that can meet most of the astronomical, instrument and weather constraints''}.

\section{Objectives}
\label{sec:objectives}
The proposed work aims to provide a fully functional algorithm that can be used in the astronomical scheduling problem in ALMA observatory.
The following are the general objectives to accomplish at the end of this work:

\begin{enumerate}

\item To develop a new scheduling algorithm that meets the ALMA’s scientific, instrument and
weather constraints, and requirements; and successfully integrate it within ALMA software.
The development will be focused in try to solve the long-term scheduling problem, where
the algorithm shall provide a schedule for different array configurations within an
observing season.

The specific goals for this objective are:
\begin{itemize}
\item Study real world constraints gathered from ALMA scheduling requirements.
\item Implement the algorithm in a suitable programming language.
\item Design a model to be applied to the short-term, medium-term and long-term ALMA’s
scheduling problem. Focusing in the long-term problem.
\item Define and develop an algorithm that satisfy the ALMA’s scheduling problem requirements
studied prior.
\item Minimize array configuration idle time, at the same time maximize the number completed observation projects.
\end{itemize}
\item Define metrics to evaluate the algorithm performance in terms of idle-time or instrument
usage, scientific throughput and completed projects.

The specific goals for this objective are:
\begin{itemize}
\item Metrics must consider idle observatory time and executive time fairness.
\item Develop a suitable test-bench software to evaluate the algorithm.
\item Validate the algorithm using the test-bench.
\end{itemize}
\end{enumerate}

\section{Expected results and validation}
The main contribution expected is to provide a model and a algorithm to solve the multiple array configuration problem and the implemented new algorithm which will be used as part of the ALMA software for the next years.
The specific expected results are:
\begin{itemize}
\item An implemented algorithm ready to use in ALMA
\item A suitable test bench to benchmark the performance of the algorithm
\item The output values of the tests done during the validation process. Also the comparison with
another algorithms whenever is possible
\end{itemize}

Since this work mainly considers the performance of an algorithm implementation, the algorithm will be validated
executing performance measurements tests and then comparing these results against previous
studies. This work proposes a new methodology to validate the performance of an observatory
scheduling algorithm hence a full comparison with previous studies could not apply.

The performance tests will be carried over a simulation platform which has been already
developed for ALMA software. The observation projects considered for the test will be real
projects taken from the ALMA’s cycle 2 observing season, however these projects are not publicly
available after a period of time (1 year).

Even when the projects are not publicly available, this shall not affect the thesis results due the
results does not show in detail what project have more chance to be completely observed, the
restriction imposed by ALMA is related to the ability to identify a particularly project and this is
not the purpose of the results that this work will present.

\section{Thesis structure}
This thesis is divided into 4 chapters: Chapter 2 discusses the relevant state of the art, for scheduling methods and providing an overview of some of the current schedulers used in observatories. Chapter 3 discusses the target problem to solve, describes necessary concepts to understand the problem, focusing in the ALMA observatory. Chapter 4 introduces the mathematical description of the solution given to the problem, and the details of the algorithms to be used to solve the problem. Chapter 5 presents the validation for the proposed algorithms. Finally, chapter 6 presents the conclusions of the work and some future work.