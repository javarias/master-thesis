\chapter{Experiments and Results}

The proposed algorithms are validated through a simulation to solve a
multiple-array instance of the ALMA scheduling problem, comparing the results obtained with the DSA when applicable.
These algorithms are tested using as input real ALMA data, intended to be used in observations for the ALMA's cycle 2.

The DSA does not optimize the allocated time for each executive along the observing season. For this reason, the DSA and the proposed algorithms are compared using the executive balance time, and the project completion rate. This is done only when the ``Astronomical observation scheduling'' problem is simulated.

The ``Array configuration planning'' is a new problem and there is no methodology on the literature to compare how it behave. The results show the feasibility of the proposed algorithm to solve the problem.

\section{Experiments setup}

The experiment consisted in a simulation of the ALMA's cycle 2 observing season. The simulator used is a modified version of the ALMA's scheduling planning simulator (see appendix~\ref{sec:implementation}). Which is setup with the input data loaded for the observation projects, weather pattern and executives expected time balance. However, it is run in two ways:
\begin{itemize}
\item The simulator is setup to load the array configuration data with fixed start and duration, and ran twice using the DSA and DRR, respectively as the scheduling algorithm.

\item A utility is run to determine the array configuration's ``SB critical set'' based on the available configurations given as input, passing these results to the SA algorithm, which calculates the start time and duration of each array configuration. Then, the simulator is run several times, while the SA tries different array configuration setup, until a ``valid'' solution is found.
\end{itemize}

The metric to compare both algorithm will be the percentage of time given to each executive every simulated week, and for the whole array configuration duration. At the same time, it will be analyzed the impact of the DRR algorithm for the project completion rate at the end of the whole simulation, checking if there is any adverse effect over this parameter. The completion rate will be based at project level, accounting projects completed, started but non-completed and non-started, these categories will be broken down based on the projects science grade given: $A$, $B$ and $C$.

To verify that the algorithm is able to produce valid solutions for the Array configuration planning, different instances of the problem are considered to show how the algorithms behave related to when difficulty of the problem increases. Each problem instance is defined as the number of A-graded projects that the algorithm is able complete. The idea behind this is to know how many iterations, in average (due the stochastic nature of the algorithm design), require to get a feasible result for that problem instance.

All the tests were run on a 2 x Intel Xeon X5675 running at 3.06 $[GHz]$, with 48 $[GiB]$ of RAM DDR3 1333 $[MHz]$, configured in 6 channels (2 x QPI 6.4 $[GT/s]$). Running Red Hat Enterprise Linux 5.9 (x86 PAE) and Oracle's Java (JRE) 1.7.0\_51.

\subsection {Input Data}
\label{sec:input-data}
This section summarizes the data that will be used to validate the solutions. 

\subsubsection{Observing Season and executives time share}
The executive time share configuration is presented in table~\ref{table:input-executive}

\begin{table}[h!]
\begin{center}
\begin{tabular}{|c|c|}
\hline
\textbf{Executive} & \textbf{Time (\%)}\\ \hline
CL & $9.5\%$ \\ \hline
EA & $21.375\%$ \\ \hline
EA\_NA & $0.0\%$\tablefootnote{For accounting purposes, the used time is split in equal parts between EA \& NA} \\ \hline
EU & $32.0625\%$ \\ \hline
NA & $32.0625\%$ \\ \hline
OTHER & $5.0\%$ \\ \hline
\end{tabular}
\end{center}
\caption{Time share partition given to each Executive}
\label{table:input-executive}
\end{table}

The following are the details for the observation season:

\begin{description}
\item[Start date]: June 1st, 2014 00:00 UTC 
\item[End date]: October 1st, 2015 00:00 UTC
\item[Start hour daily observation interval]: 23:00 UTC
\item[End date daily observation interval]: 09:00 UTC
\item[Total number of hours available for science]: $4861\,[h]$
\end{description}

\subsubsection{Array Configurations}

The fixed array configuration used to compare DSA vs DRR is presented in table~\ref{table:drr-test-array-config}

\begin{table}[h!]
\begin{center}
\begin{tabular}{|c|c|c|}
\hline
\textbf{Array configuration} & \textbf{Start date (UTC)} & \textbf{End date (UTC)} \\ \hline
C34-6 & 2014-06-01 & 2014-07-14 \\ \hline
C34-7 & 2014-07-15 & 2014-08-31 \\ \hline
C34-6 & 2014-12-01 & 2014-12-31 \\ \hline
C34-5 & 2015-01-01 & 2015-01-31 \\ \hline
C34-4 & 2015-03-01 & 2015-03-31 \\ \hline
C34-3 & 2015-04-01 & 2015-04-30 \\ \hline
C34-2 & 2015-05-01 & 2015-05-31 \\ \hline
C34-1 & 2015-06-01 & 2015-06-30 \\ \hline
C34-4 & 2015-07-01 & 2015-07-31 \\ \hline
C34-6 & 2015-08-01 & 2015-08-30 \\ \hline
C34-7 & 2015-09-01 & 2015-10-01 \\ \hline
7m & 2014-06-01 & 2015-10-01 \\ \hline
TP & 2014-06-01 & 2015-10-01 \\ \hline
\end{tabular}
\end{center}
\caption{Array configuration plan used to compare DRR algorithm vs DSA algorithm}
\label{table:drr-test-array-config}
\end{table}

For ALMA' s cycle 2, the list of the array configurations is presented in table~\ref{table:input-array-configs}. Although the table does not introduce the starting and ending dates for any of the configurations, which are part of the problems presented in this work, and solved by the algorithms introduced in this document.

\begin{table}[h!]
\begin{center}
\begin{tabular}{|c|c|c|c|}
\hline
\textbf{Configuration name} & \textbf{Array type} & \textbf{Min. Baseline $[m]$} & \textbf{Max. Baseline $[m]$}\\
\hline
C34-1 & $12\,[m]$ & $14.2$ & $165.6$ \\
\hline
C34-2 & $12\,[m]$ & $14.1$ & $303.6$ \\
\hline
C34-3 & $12\,[m]$ & $20.6$ & $442.7$ \\
\hline
C34-4 & $12\,[m]$ & $20.6$ & $558.2$ \\
\hline
C34-5 & $12\,[m]$ & $25.8$ & $820.2$ \\
\hline
C34-6 & $12\,[m]$ & $40.6$ & $1091.0$ \\
\hline
C34-7 & $12\,[m]$ & $40.6$ & $1507.9$ \\
\hline
7m    & $7\,[m]$  & $8.9$  & $32.1$ \\
\hline
TP    & $TP$      & $-$    & $-$ \\
\hline
\end{tabular}
\end{center}
\caption{Array configurations provided as input for the algorithms}
\label{table:input-array-configs}
\end{table}

\subsubsection{Observation projects}
As input for the validation tests and benchmarking of the software developed in this work, there will be used the cycle 2 proposals submitted to ALMA observatory, although these proposals have been already processed and converted into APDM Observation projects and APDM Scheduling Blocks, the data used for tests is an extract of them, this data contains only the relevant parts for the scheduling simulation.

At the moment when this document was written, the Observation Projects are not still publicly available, but according to ALMA observatory policy, they will become available at the end of 2014 or beginning of 2015. 

A summary of the Scheduling blocks that the validation will be used is presented in table~\ref{table:requested-time-12m} for $12\,[m]$ array configurations, in table~\ref{table:requested-time-7m} for $7\,[m]$ array configurations and in table~\ref{table:requested-time-tp} for $TP$  array configurations. Each table have in detail the number of hours requested for each executive, broken down the time per science grade. $D$ graded projects are not shown in the tables, although they are available as part of the data set used as input.

\begin{table}
\begin{center}
\begin{tabular}{|c|c|c|c|}
\hline
\textbf{Grade} & \textbf{Executive} & \textbf{Time Requested $[h]$} & \textbf{Number of Scheduling Blocks} \\ \hline
A &	CL		& 12.0  & 2 \\ \hline
A &	EA		& 36.0  & 13 \\ \hline
A &	EA/NA	& 6.0   & 3 \\ \hline
A & EU      & 112.0 & 31 \\ \hline 
A &	NA		& 114.0	& 31 \\ \hline
A & OTHER	& 0		& 0 \\ \hline
B  & CL 	& 160.0		& 48  \\ \hline
B  & EA     & 438.0     & 145 \\ \hline
B  & EA/NA  & 106.0     & 30  \\ \hline
B  & EU     & 828.0     & 280 \\ \hline
B  & NA     & 1024.0    & 344 \\ \hline
B  & OTHER  & 40.0      & 19  \\ \hline
C  & CL     & 130.0     & 50  \\ \hline
C  & EA     & 246.0     & 65  \\ \hline
C  & EA/NA  & 52.0      & 23  \\ \hline
C  & EU     & 410.0     & 155 \\ \hline
C  & NA     & 556.0     & 196 \\ \hline
C  & OTHER  & 34.0      & 13  \\ \hline
\end{tabular}
\end{center}
\caption{Requested time for $12m$ array configurations}
\label{table:requested-time-12m}
\end{table}

\begin{table}
\begin{center}
\begin{tabular}{|c|c|c|c|}
\hline
\textbf{Grade} & \textbf{Executive} & \textbf{Time Requested $[h]$} & \textbf{Number of Scheduling Blocks} \\ \hline
A &	CL		& 0  & 0 \\ \hline
A &	EA		& 24.0  & 1 \\ \hline
A &	EA/NA	& 6.0   & 12 \\ \hline
A & EU      & 12.0 & 1 \\ \hline 
A &	NA		& 16.0 & 2 \\ \hline
A & OTHER	& 0		& 0 \\ \hline
B  & CL 	& 16.0		& 1  \\ \hline
B  & EA     & 78.0     & 21 \\ \hline
B  & EA/NA  & 22.0     & 5  \\ \hline
B  & EU     & 168.0     & 41 \\ \hline
B  & NA     & 182.0    & 29 \\ \hline
B  & OTHER  & 0.0      & 0  \\ \hline
C  & CL     & 38.0     & 2  \\ \hline
C  & EA     & 46.0     & 12  \\ \hline
C  & EA/NA  & 36.0      & 9  \\ \hline
C  & EU     & 5800     & 7 \\ \hline
C  & NA     & 556.0     & 196 \\ \hline
C  & OTHER  & 72.0      & 12  \\ \hline
\end{tabular}
\end{center}
\caption{Requested time for $7m$ array configurations}
\label{table:requested-time-7m}
\end{table}

\begin{table}
\begin{center}
\begin{tabular}{|c|c|c|c|}
\hline
\textbf{Grade} & \textbf{Executive} & \textbf{Time Requested $[h]$} & \textbf{Number of Scheduling Blocks} \\ \hline
A &	CL		& 0  & 0 \\ \hline
A &	EA		& 28.0  & 1 \\ \hline
A &	EA/NA	& 208.0   & 2 \\ \hline
A & EU      & 132.0 & 1 \\ \hline 
A &	NA		& 112.0 & 2 \\ \hline
A & OTHER	& 0		& 0 \\ \hline
B  & CL 	& 28.0		& 1  \\ \hline
B  & EA     & 1852.0     & 21 \\ \hline
B  & EA/NA  & 752.0     & 5  \\ \hline
B  & EU     & 922.0     & 41 \\ \hline
B  & NA     & 1792.0    & 29 \\ \hline
B  & OTHER  & 0.0      & 0  \\ \hline
C  & CL     & 52.0     & 2  \\ \hline
C  & EA     & 1250.0     & 12  \\ \hline
C  & EA/NA  & 824.0      & 9  \\ \hline
C  & EU     & 368.0     & 7 \\ \hline
C  & NA     & 78.0     & 12 \\ \hline
C  & OTHER  & 2.0      & 1  \\ \hline
\end{tabular}
\end{center}
\caption{Requested time for $TP$ array configurations}
\label{table:requested-time-tp}
\end{table}

