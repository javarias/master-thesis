\thispagestyle{empty}
\vspace*{\fill}

\selectlanguage{english}
\begin{center}
\begin{LARGE}\textbf{Abstract}\end{LARGE}
\end{center}

The Atacama Large Millimeter Array (ALMA) is the biggest astronomical project, currently under
construction in the Chilean Atacama desert. This radio-telescope consists of 66
antennas, the telescope will be able to observe simultaneously different sky sources, organized in one or more
variable size arrays. The observatory's scheduling system considers full automatic
operation, and dynamic rescheduling of the observations according to factors changing continuously, like atmospheric
conditions, source visibility, technical failures or targets of opportunity.
Currently, ALMA, has a dynamic scheduling algorithm which considers a wide-range of scientific,
instrument and weather constraints and requirements, but multiples issues have been
identified. Newer models provide a fresh perspective of this problem, but they lack of the
scientific background, and they have not been tested in a real world problem.
In this thesis will be proven that, is possible to design and implement a new algorithm for ALMA observatory 
considering the long-term scheduling problem, for what the algorithm will provide a schedule 
for different array configurations within an observing
season. The algorithm will be verified and validated using real data, trying to define a metric based on quality
of the scientific output and instrument usage.

\vspace{0.5cm}

\selectlanguage{spanish}
\begin{center}
\begin{LARGE}\textbf{Resumen}\end{LARGE}
\end{center}

El Atacama Large Millimeter Array (ALMA) es el mayor proyecto astron\'omico actualmente en
construcci\'on en el desierto chileno de Atacama. Este radio-telescopio consiste en 66
antenas, que ser\'an capaz de observar simult\'aneamente varias fuentes en el cielo, organizado en una o m\'as arreglos
de tama\~no variable. El sistema de planificaci\'on de observaciones del telescopio
considera una operaci\'on totalmente autom\'atica, y una reprogramaci\'on din\'amica de las tareas de
acuerdo a los factores cambiantes, como las condiciones atmosf\'ericas, visibilidad de la fuente,
fallas t\'ecnicas u objetivos de oportunidad.
Actualmente ALMA tiene un algoritmo de planificaci\'on din\'amica que considera una amplia
gama de restricciones y requerimientos del instrumento cient\'ifico y del clima, sin embargo
m\'ultiples problemas han sido identificados. Los modelos m\'as nuevos ofrecen una nueva
perspectiva de este problema, pero carecen de un trasfondo cient\'ifico y no han sido probados en
un problema del mundo real. Esta tesis probar\'a que es posible dise\~nar e implmentar un nuevo algoritmo para ALMA
teniendo en cuenta el problema de planificaci\'on a largo plazo, donde el algoritmo presentar\'a
un cronograma para las diferentes configuraciones de arreglo disponibles durante una misma
temporada de observaci\'on. El algoritmo se verificar\'a con datos reales, tratando de definir una
m\'etrica basada en la calidad de la producci\'on cient\'ifica y el uso de instrumentos.
\vfill

\cleardoublepage